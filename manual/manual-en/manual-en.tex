\documentclass[12pt,english]{extarticle}
\usepackage{mathptmx}
\usepackage[T1]{fontenc}
%\usepackage[latin9]{inputenc}
\usepackage[letterpaper]{geometry}
\geometry{verbose,tmargin=2.0cm,bmargin=2.0cm,lmargin=2.5cm,rmargin=2.5cm,headheight=1.5cm,headsep=1.5cm,footskip=1.5cm}
\setcounter{tocdepth}{2}
\setlength{\parskip}{\smallskipamount}
\setlength{\parindent}{0pt}
\usepackage{color}
\usepackage{prettyref}
\usepackage{float}
\usepackage{amsmath}
\usepackage{graphicx}
\usepackage{subfigure}
\usepackage{amssymb}
\usepackage{fancyvrb}
\usepackage{hyperref}
\usepackage{babel}
\usepackage{makeidx}

\usepackage{listings,xcolor}
\lstset{language={[77]Fortran},
        extendedchars=false,          % non-English characters
        lineskip=2pt,
        backgroundcolor=\color{white},
        basicstyle=\tt\scriptsize\color{black},
        commentstyle=\tt\color{green!40!black},
        keywordstyle=\tt\color{blue},
        stringstyle=\tt\color{magenta},
        showspaces=false,             % underline spaces in the codes
        showstringspaces=false,       % underline spaces only in a string
        showtabs=false,               % underline tabs in the codes
        identifierstyle=\tt\color{red!60!black},
        numberstyle=\tiny\color{black},
        numbersep=14pt,               % how far the line-numbers are from the code
        texcl=false,                  % comments in LaTeX if true
        emph={                        % keywords to be highlighted
%               subroutine, return,
%               end
        },
        emphstyle=\sf\bfseries\color{red!50!black}\fcolorbox{orange!40!white}{.},
        numbers=left,
        rulecolor=\color{green},
        frame=single
%        tabsize=2,                    % set default tab-size to 2 spaces
%        frame=shadowbox, rulesepcolor=\color{blue}
}

\makeindex

\begin{document}

\title{User's Guide of the Program \textsc{UniMoVib} \\ \vspace{10 mm} (Ver. 1.0.1)  \vspace{70 mm}}

\date{\today}

\author{Wenli Zou \\ \vspace{5mm}
\href{mailto:qcband@gmail.com}{qcband@gmail.com} \\ \vspace{7cm}}

\maketitle
\setcounter{page}{0}
\thispagestyle{empty}

\begin{center}
\emph{Institute of Modern Physics, Northwest University, Xi'an, China}
\end{center}

\pagebreak{}

\tableofcontents{}

\pagebreak{}

\section{About \textsc{UniMoVib}} \label{part:about}

\textsc{UniMoVib} is a \textsc{Uni}fied interface for \textsc{Mo}lecular \textsc{Vib}rational harmonic frequency calculations. It was initially written in Fortran 77 during 2014 and 2015 when I was in Dallas, US and in Tianjin, China, and worked as a front-end interface for the \textsc{Localmodes} program developed in the \href{https://sites.smu.edu/dedman/catco/}{CATCO group} in SMU, but its ancient predecessor has been started since 2009 when I was in UT Austin. After being rewritten in Fortran 90 recently, \textsc{UniMoVib} is released as a stand-alone program.

\subsection{Features} \label{sec:feature}
\index{{Features}@{Features}}

\begin{itemize}
\item Calculate harmonic vibrational frequencies and (optional) I.R. intensities from Hessian, coordinates, and other related data generated by quantum chemistry programs or by the user manually. \\
    Nearly 30 quantum chemistry programs have been supported (see Section \ref{sec:inp-contrl}), although some of them can do these calculations much better.
\index{{Quantum chemistry program}@{Quantum chemistry program}!MOPAC}
\item Analyze point group of geometry and irreducible representations (\emph{irreps.}) of normal modes in full symmetry (for \emph{irreps.}: closed-shell molecule only). \\
    Symmetry analysis is based on the symmetry subroutines from \textsc{Mopac} 7.1, which is fully in the public domain (at \href{http://openmopac.net/Downloads/Downloads.html}{openmopac.net} and \href{https://sourceforge.net/projects/mopac7/}{sourceforge.net}).
\item Thermochemistry calculation uses the point group in full symmetry, and the results are printed in Gaussian-style (for detailed explanations, see Foresman and Frisch, \emph{Exploring Chemistry With Electronic Structure Methods}, Ed.2, Gaussian Inc., Pittsburgh, PA, \textbf{1996}, pp.66).
\item Save a \textsc{Molden} file for animation of normal modes.
\item Set up isotopic masses, temperature, pressure, scale factor and/or experimental frequencies, and so on.
\item Can be used as a third party module for frequency and thermochemistry calculations in a quantum chemistry program, especially when non-Abelian group is not supported therein.
\end{itemize}

\subsection{Symmetry} \label{sec:symm}
\index{{Symmetry}@{Symmetry}}

The point group symmetries supported are listed in Table \ref{tab:symm}.

\begin{table}[H]
\caption{Available Point Groups.}\label{tab:symm}
\small\centering
\begin{tabular}{ll}
\hline\hline
$C_{n}$           & n = 1\ldots 8 \\
$C_s$, $C_{nv}$   & n = 2\ldots 8 \\
$C_i$, $C_{nh}$   & n = 2\ldots 8 \\
$D_{n}$           & n = 2\ldots 8 \\
$D_{nd}$          & n = 2\ldots 7 \\
$D_{nh}$          & n = 2\ldots 8 \\
$S_{n}$           & n = 4, 6, 8 \\
Others            & $R_3$, $T$, $T_d$, $T_h$, $O$, $O_h$, $I$, $I_h$, $C_{\infty v}$, $D_{\infty h}$ \\
\hline\hline
\end{tabular}
\end{table}

Two point groups will be printed by the program, \emph{i.e.} "\verb|Point Group (Z)|" and "\verb|Point Group (Z+M)|". \verb|Point Group (Z)| is independent of isotopic masses whereas \verb|Point Group (Z+M)| is. Isotopes may lead to lower symmetry of vibrational normal modes, so the latter point group should be used to analyze vibrations and do thermochemistry calculations. However, some quantum chemistry programs use the former symmetry, which is not correct, and therefore the \emph{irreps.} cannot be correctly analyzed. A extreme case is the fullerence $^{12}$C$_{59}{}^{13}$C, which has $I_h$ symmetry for geometry and electronic structure, but $C_1$ for vibration and rotation. If $I_h$ is used in thermochemistry calculations, the error in Gibbs free energy will be as large as 2.5 kcal/mol!

\pagebreak{}


\section{Compiling abd Running} \label{part:setting}

\subsection{Compiling the program} \label{sec:install}
\index{{Installation}@{Installation}}

\verb|$ cd $UniMoVib/src| \\
\verb|$ make |

A Fortran90 compiler is required, which is defined in \verb|Makefile|.


\subsection{Running the program} \label{sec:run}
\index{{Running}@{Running}}

Double-click the binary program \verb|unimovib.exe|, and type in the name of input file (MS-Windows only), \\ \\
or \\
in the terminal, type in \\
\verb|$ ./unimovib.exe | \\
and then type in the name of input file (if no input file name provided, the default name \verb|job.inp| will be assumed), \\ \\
or \\
in the terminal, type in \\
\verb|$ ./unimovib.exe -b < input > output | \\

In the last way, one can prepare a batch script to perform a series of calculations.

\pagebreak{}


\section{Input description} \label{part:input}

The input options are grouped by namelists, which are ended by \verb|$END|.
These groups may be given in any order as desired. Before each \verb|$| symbol
there should be at least one space.

The input file is case-insensitive except the data file names specified in the \verb|$QCData|
group (section \ref{sec:inp-qcdata}).

\subsection{\texttt{\$Contrl} group} \label{sec:inp-contrl}
\index{{\texttt{\$Contrl} group}@{\texttt{\$Contrl} group}}

This group specifies the type of calculation. Keywords:

\bigskip{}\bigskip{}
\index{{\texttt{\$Contrl} group}@{\texttt{\$Contrl} group}!\texttt{QCProg}}
\index{{Quantum chemistry program}@{Quantum chemistry program}}
\verb|QCProg="XXXX"|: \verb|XXXX| is the name of quantum chemistry program that
is used to calculate the molecular Hessian matrix and vibrational frequencies. The
following programs are supported:
\begin{itemize}
\index{{Quantum chemistry program}@{Quantum chemistry program}!Gaussian}
\item \verb|Gaussian| (default).
\index{{Quantum chemistry program}@{Quantum chemistry program}!GAMESS}
\item \verb|GAMESS| (\verb|GAMESS-US| and \verb|GAMESSUS| are synonyms).
\index{{Quantum chemistry program}@{Quantum chemistry program}!Firefly}
\item \verb|Firefly| (\verb|PCGamess| and \verb|PC-Gamess| are synonyms).
\index{{Quantum chemistry program}@{Quantum chemistry program}!GAMESS-UK}
\item \verb|GAMESS-UK| (\verb|GAMESSUK| is synonym).
\index{{Quantum chemistry program}@{Quantum chemistry program}!ORCA}
\item \verb|ORCA|.
\index{{Quantum chemistry program}@{Quantum chemistry program}!Molpro}
\item \verb|Molpro|.
\index{{Quantum chemistry program}@{Quantum chemistry program}!Q-Chem}
\item \verb|QChem| (\verb|Q-Chem| is synonym).
\index{{Quantum chemistry program}@{Quantum chemistry program}!NWChem}
\item \verb|NWChem|.
\index{{Quantum chemistry program}@{Quantum chemistry program}!CFour}
\item \verb|CFour|.
\index{{Quantum chemistry program}@{Quantum chemistry program}!Turbomole}
\item \verb|Turbomole|.
\index{{Quantum chemistry program}@{Quantum chemistry program}!deMon2k}
\item \verb|deMon2k| (\verb|deMon| is synonym).
\index{{Quantum chemistry program}@{Quantum chemistry program}!PQS}
\item \verb|PQS|.
\index{{Quantum chemistry program}@{Quantum chemistry program}!MOPAC}
\item \verb|OpenMOPAC| (\verb|MOPAC| is synonym). \textsc{Mopac} 6 and \textsc{Mopac} 7 are also supported, but
the closely related \textsc{Fujitsu Mopac} 200x (now MO-G in \textsc{Scigress}) has not been tested.
\index{{Quantum chemistry program}@{Quantum chemistry program}!AMSOL}
\index{{Quantum chemistry program}@{Quantum chemistry program}!AMPAC}
\item \verb|AMSOL| (\verb|AMPAC| is synonym). \textsc{Ampac} 2.x is also supported, but the higher
versions of \textsc{Ampac} have not been tested.
\index{{Quantum chemistry program}@{Quantum chemistry program}!Dalton}
\item \verb|Dalton|.
\index{{Quantum chemistry program}@{Quantum chemistry program}!FHI-AIMS}
\item \verb|FHI-AIMS| (\verb|FHIAIMS| and \verb|AIMS| are synonyms).
\index{{Quantum chemistry program}@{Quantum chemistry program}!CP2k}
\item \verb|CP2k|. The \textsc{QuickStep} module.
\index{{Quantum chemistry program}@{Quantum chemistry program}!Hyperchem}
\item \verb|Hyperchem|.
\index{{Quantum chemistry program}@{Quantum chemistry program}!Jaguar}
\item \verb|Jaguar|. A quantum chemistry module in \textsc{Schr\"odinger Suite}.
\index{{Quantum chemistry program}@{Quantum chemistry program}!ADF}
\item \verb|ADF|. Only the molecular \textsc{Adf} module was tested.
\index{{Quantum chemistry program}@{Quantum chemistry program}!MOLDEN}
\index{{Quantum chemistry program}@{Quantum chemistry program}!ACES}
\index{{Quantum chemistry program}@{Quantum chemistry program}!COLUMBUS}
\index{{Quantum chemistry program}@{Quantum chemistry program}!MOLCAS}
\index{{Quantum chemistry program}@{Quantum chemistry program}!Dalton}
\item \verb|MOLDEN|, which was generated by a frequency calculation and there should be at
least three sections: \verb|[FREQ]|, \verb|[FR-COORD]|, and \verb|[FR-NORM-COORD]| in it. Through the
\textsc{Molden} file, \textsc{Aces-II}, \textsc{Columbus}, \textsc{Dalton} (analytic frequency calculation), \textsc{Molcas}, and so on may be supported by the program.
\index{{Quantum chemistry program}@{Quantum chemistry program}!Crystal}
\item \verb|Crystal|. Molecular harmonic frequency is supported and \textsc{Crystal} 14 has been tested.
\index{{Quantum chemistry program}@{Quantum chemistry program}!Spartan}
\item \verb|Spartan|.
\index{{Quantum chemistry program}@{Quantum chemistry program}!PSI}
\item \verb|PSI|. Only \textsc{Psi} 4 has been tested.
\index{{Quantum chemistry program}@{Quantum chemistry program}!DMOL3}
\item \verb|DMOL3| (\verb|DMOL| is synonym). Molecular harmonic frequency is supported.
\index{{Quantum chemistry program}@{Quantum chemistry program}!ACES}
\item \verb|ACES|. Both \textsc{Aces-II} and \textsc{Aces-III} have been tested.
\index{{UniMoVib format}@{UniMoVib format}}
\item \verb|UniMoVib| (\verb|ALM| is synonym). A plain text file generated by the \textsc{UniMoVib} program. See Appendix \ref{sec:almfmt}.
\end{itemize}
\index{{XYZ format}@{XYZ format}}
\verb|     |In addition, \verb|QCProg="AtomCalc"| will do an atomic thermochemistry calculation (see section \ref{sec:inp-atom}).

\bigskip{}\bigskip{}
\index{{\texttt{\$Contrl} group}@{\texttt{\$Contrl} group}!\texttt{IFConc}}
\verb|IFConc|: (\verb|.True.|/\verb|.False.|) Concise output of frequencies or not. Default: \verb|.False.|

\bigskip{}\bigskip{}
\index{{\texttt{\$Contrl} group}@{\texttt{\$Contrl} group}!\texttt{Isotop}}
\index{{\texttt{\$IsoMas} group}@{\texttt{\$IsoMas} group}}
\index{{Isotopic mass}@{Isotopic mass}}
\verb|Isotop|: Sets up isotopic masses.
\begin{description}
\item[ ]\verb|  = 0|: (default) all the atomic masses will be read from the data file of frequency calculation;
if there was none, then the masses are taken from the library (the most abundant isotopic masses will be used except for several quantum chemistry programs).
\item[ ]\verb|  = 1|: all the atomic masses will be read from library or the data file of
frequency calculation (same as \verb|0|), and then the masses of a list of isotopes will be
replaced by the values provided after the \verb|$IsoMas| group (section \ref{sec:inp-isomas}).
\item[ ]\verb|  = 2|: all the atomic masses are provided after the \verb|$IsoMas| group (section \ref{sec:inp-isomas}).
\end{description}

\bigskip{}\bigskip{}
\index{{\texttt{\$Contrl} group}@{\texttt{\$Contrl} group}!\texttt{ISyTol}}
\index{{Symmetry}@{Symmetry}}
\verb|ISyTol = MN|: the symmetry tolerance is defined by $M*10^{N-3}$ where M
is always positive and the sign of \verb|ISyTol| will pass to N. So
\verb|ISyTol = 21| means 0.02 whereas \verb|-21| means 0.0002. Default: 10, \emph{i.e.} the tolerance is 0.001.

\bigskip{}\bigskip{}
\index{{\texttt{\$Contrl} group}@{\texttt{\$Contrl} group}!\texttt{IFExp}}
\index{{\texttt{\$ExpFrq} group}@{\texttt{\$ExpFrq} group}}
\index{{Experimental frequency correction}@{Experimental frequency correction}}
\verb|IFExp|: (\verb|.True.|/\verb|.False.|) Correct the Hessian matrix using
experimental vibrational frequencies which are provided after the \verb|$ExpFrq| group (section \ref{sec:inp-expfrq}).
Default: \verb|.False.|

\bigskip{}\bigskip{}
\index{{\texttt{\$Contrl} group}@{\texttt{\$Contrl} group}!\texttt{IFSAVE}}
\verb|IFSAVE|: (\verb|.True.|/\verb|.False.|) Save the atomic masses (affected by
\verb|Isotop|), Cartesian coordinates, Hessian matrix (affected by
\verb|IFExp|), and the APT matrix into a plain text data file *.umv. This option
doesn't work with \verb|QCProg="AtomCalc"| or \verb|"UniMoVib"|. Default: \verb|.False.| See Appendix \ref{sec:almfmt} for the format.

\bigskip{}\bigskip{}
\index{{\texttt{\$Contrl} group}@{\texttt{\$Contrl} group}!\texttt{IFMOLDEN}}
\index{{Quantum chemistry program}@{Quantum chemistry program}!MOLDEN}
\index{{Quantum chemistry program}@{Quantum chemistry program}!Gabedit}
\verb|IFMOLDEN|: (\verb|.True.|/\verb|.False.|) Save a \textsc{Molden} file except when \verb|QCProg="AtomCalc"| or \verb|"MOLDEN"|, which may be opened by the \textsc{Molden} or \textsc{Gabedit} program to view geometry and normal vibration modes. Default: \verb|.False.|


\subsection{\texttt{\$QCData} group} \label{sec:inp-qcdata}
\index{{\texttt{\$QCData} group}@{\texttt{\$QCData} group}}
\index{{Quantum chemistry program}@{Quantum chemistry program}}

This group specifies data file(s) enclosed by quotes, where the data (atomic
masses, coordinates, APT, and Hessian) are obtained. In general,
only one data file is required, which is defined by the option
\verb|FCHK|. However for some programs, multiple data files should be
defined separately by the keywords \verb|HESS|, \verb|DDIP|, and/or
\verb|GEOM|.

\index{{\texttt{\$QCData} group}@{\texttt{\$QCData} group}!\texttt{FCHK}}
\index{{\texttt{\$QCData} group}@{\texttt{\$QCData} group}!\texttt{HESS}}
\index{{\texttt{\$QCData} group}@{\texttt{\$QCData} group}!\texttt{DDIP}}
\index{{\texttt{\$QCData} group}@{\texttt{\$QCData} group}!\texttt{GEOM}}

\begin{itemize}
\index{{Quantum chemistry program}@{Quantum chemistry program}!Gaussian}
\item \textsc{Gaussian}: *.fchk. By default, the atomic masses are not included in the fchk
file, so the most abundant isotopic masses are assumed, but for \textsc{Gaussian} 09
(and maybe higher versions in the future), one can also use \texttt{FREQ(SaveNormalModes)} instead
to save atomic masses automatically.
\index{{Quantum chemistry program}@{Quantum chemistry program}!GAMESS}
\item \textsc{Gamess}: *.dat (by \verb|FCHK|) + *.out (by \verb|GEOM|).
\index{{Quantum chemistry program}@{Quantum chemistry program}!Firefly}
\item \textsc{Firefly}: data file (by \verb|FCHK|; default name: \verb|PUNCH|) + *.out (by \verb|GEOM|).
\index{{Quantum chemistry program}@{Quantum chemistry program}!GAMESS-UK}
\item \textsc{Gamess-uk}: *.out file. Use \texttt{RUNTYPE INFRARED} in the frequency calculation to
print APT if you are interested in the IR intensities.
\index{{Quantum chemistry program}@{Quantum chemistry program}!ORCA}
\item \textsc{Orca}: *.hess.
\index{{Quantum chemistry program}@{Quantum chemistry program}!Molpro}
\item \textsc{Molpro}: *.out file. Use the following commands to print Hessian and APT: \\
\verb|{frequencies,print=1;print,hessian}|
\index{{Quantum chemistry program}@{Quantum chemistry program}!Q-Chem}
\item \textsc{Q-Chem}: *.fchk. In your \textsc{Q-Chem} frequency calculation, use \texttt{GUI=2} to
generate the *.fchk file. The atomic masses are not included in the fchk
file, so the most abundant isotopic masses are assumed.
\index{{Quantum chemistry program}@{Quantum chemistry program}!NWChem}
\item \textsc{NWChem}: *.out file (by \verb|FCHK|) + *.fd{\_}ddipole (by \verb|DDIP|) +
*.hess (by \verb|HESS|), where \verb|DDIP| is optional and can be
neglected if you are not interested in the IR intensities.
\index{{Quantum chemistry program}@{Quantum chemistry program}!CFour}
\item \textsc{CFour}
  \begin{description}
  \item[ ]For analytical frequency (\texttt{VIB=ANALYTIC}): *.out file (by \verb|FCHK|) + GRD
  (by \verb|GEOM|).
  \index{{Quantum chemistry program}@{Quantum chemistry program}!MOLDEN}
  \item[ ]For both numerical frequency and analytical frequency: Use the \textsc{Molden} file. However, no IR intensities.
  See also \texttt{MOLDEN} below.
  \end{description}
\index{{Quantum chemistry program}@{Quantum chemistry program}!Turbomole}
\item \textsc{Turbomole}: *.out file of aoforce (by \verb|FCHK|; default: aoforce.out) +
dipgrad (by \verb|DDIP|), where
\verb|DDIP| is optional and can be neglected if you are not interested in
the IR intensities.
\index{{Quantum chemistry program}@{Quantum chemistry program}!deMon2k}
\item \textsc{deMon2k}: *.out file (by \verb|FCHK|; default: deMon.out). \textsc{deMon2k} can print Hessian
by \texttt{PRINT DE2}.
\index{{Quantum chemistry program}@{Quantum chemistry program}!PQS}
\item \textsc{Pqs}: *.coord file (by \verb|FCHK|) + *.deriv (by \verb|DDIP|)+ *.hess
(by \verb|HESS|), where \verb|DDIP| is optional and can be neglected if
you are not interested in the IR intensities.
\index{{Quantum chemistry program}@{Quantum chemistry program}!MOPAC}
\item \textsc{OpenMopac}: *.out file (by \verb|FCHK|). Use \texttt{FORCE DFORCE} or
\texttt{FORCE=DFORCE} to print Hessian. The averaged isotopic masses
are used, which may be not consistent with some very old versions of \textsc{Mopac}.
\index{{Quantum chemistry program}@{Quantum chemistry program}!AMSOL}
\index{{Quantum chemistry program}@{Quantum chemistry program}!AMPAC}
\item \textsc{Amsol}: *.out file (by \verb|FCHK|). Use \texttt{FORCE DFORCE} to print
Hessian. The averaged isotopic masses are used, which may be not
consistent with some very old versions of \textsc{Amsol/Ampac}.
\index{{Quantum chemistry program}@{Quantum chemistry program}!Dalton}
\item \textsc{Dalton}: *.out file (by \verb|FCHK|). Since \textsc{Dalton} doesn't print nuclear
charges and element symbols, the standard element symbols have to be
specified in the input file of \textsc{Dalton}'s frequency calculation (\emph{ie.}, Mg is
okay, but Mg01 and Mgxx don't work).
\index{{Quantum chemistry program}@{Quantum chemistry program}!FHI-AIMS}
\item \textsc{Fhi-Aims}: masses.*.dat file (by \verb|FCHK|) + grad{\_}dipole.*.dat (by
\verb|DDIP|) + hessian.*.dat (by \verb|HESS|), where \verb|DDIP| is
optional and can be neglected if you are not interested in the IR
intensities.
\index{{Quantum chemistry program}@{Quantum chemistry program}!CP2k}
\item \textsc{CP2k}: the output file of frequency calculation (by \verb|FCHK|) using the
\textsc{Quickstep} module.
\index{{Quantum chemistry program}@{Quantum chemistry program}!Hyperchem}
\item \textsc{HyperChem}: the log file of frequency calculation (by \verb|FCHK|).
\textsc{HyperChem} doesn't generate the log file by default. Before doing a frequency
calculation, go to the \textbf{File} menu and select \textbf{Save log} with print level = 9 to save a
log file.
\index{{Quantum chemistry program}@{Quantum chemistry program}!Jaguar}
\item \textsc{Jaguar}: the output file of frequency calculation (by \verb|FCHK|).
\index{{Quantum chemistry program}@{Quantum chemistry program}!ADF}
\item \textsc{Adf}: the formatted TAPE21 or TAPE13 data file (by \verb|FCHK|). There are
some problems in the case of numerical frequency calculation of \textsc{Adf}. See the
Known problems section (\ref{part:problem}).
\index{{Quantum chemistry program}@{Quantum chemistry program}!MOLDEN}
\item \textsc{Molden}: a data file (by \verb|FCHK|), which contains \verb|[FREQ]|,
\verb|[FR-COORD]|, and \verb|[FR-NORM-COORD]| sections.
See the Known problems section (\ref{part:problem}).
\index{{Quantum chemistry program}@{Quantum chemistry program}!Crystal}
\item \textsc{Crystal}: the output file from molecular harmonic frequency calculation (by \verb|FCHK|).
\index{{Quantum chemistry program}@{Quantum chemistry program}!Spartan}
\item \textsc{Spartan}: the *.smol archive file (by \verb|FCHK|). The most abundant isotopic masses are assumed.
\index{{Quantum chemistry program}@{Quantum chemistry program}!PSI}
\item \textsc{Psi}: the output file (by \verb|FCHK|). In your \textsc{Psi} 4 frequency calculation, use \texttt{set print 3} to
print Hessian matrix.
\index{{Quantum chemistry program}@{Quantum chemistry program}!DMOL3}
\item \textsc{Dmol3}: the output file (by \verb|FCHK|).
\index{{Quantum chemistry program}@{Quantum chemistry program}!ACES}
\index{{Quantum chemistry program}@{Quantum chemistry program}!MOLDEN}
\item \textsc{Aces}: the output file (by \verb|FCHK|). But using the \textsc{Molden} file can achieve higher accuracy. See the Known problems section (\ref{part:problem}).
\index{{UniMoVib format}@{UniMoVib format}}
\item \textsc{UniMoVib}: an ASCII data file (by \verb|FCHK|), which was generated by \textsc{UniMoVib} with the option \verb|IFSAVE=.TRUE.|, or created manually (see Appendix \ref{sec:almfmt}).
\index{{XYZ format}@{XYZ format}}
\item XYZ: a standard XYZ data file (by \verb|FCHK|). For debugging only.
\end{itemize}

See the examples in \verb|$UniMoVib/test|.


\subsection{\texttt{\$IsoMas} group} \label{sec:inp-isomas}
\index{{\texttt{\$IsoMas} group}@{\texttt{\$IsoMas} group}}
\index{{Isotopic mass}@{Isotopic mass}}

This group is required when \verb|Isotop = 1| or \verb|2|. There is no option in
this group. After this group, the isotopic masses are provided.

\bigskip{}
If \verb|Isotop = 1|, one atom per line, including the atom index and its
mass. The program will read isotopic masses until a blank line or
the end is encountered. For example,
\begin{Verbatim}[frame=single]
 $IsoMas $End
2 15.99491
4  2.01410
\end{Verbatim}
It means that the masses of the second and the forth atoms are 15.99491 and
2.01410, respectively.

\bigskip{}
If \verb|Isotop = 2|, all the N atomic masses are defined in free format.
For example,
\begin{Verbatim}[frame=single]
 $IsoMas $End
12.0 1.0 1.0
1.0
1.0
\end{Verbatim}
5 atomic masses are defined for CH$_4$ in the above example.


\subsection{\texttt{\$ExpFrq} group} \label{sec:inp-expfrq}
\index{{\texttt{\$ExpFrq} group}@{\texttt{\$ExpFrq} group}}
\index{{Experimental frequency correction}@{Experimental frequency correction}}

This group is required when \verb|IFExp=.True.| There is only one option
\verb|MODE| in this group. After this group, the experimental vibrational
frequencies are provided.

\bigskip{}
\index{{\texttt{\$ExpFrq} group}@{\texttt{\$ExpFrq} group}!\texttt{MODE}}
If \verb|MODE = 0| (default), all the $N_{Vib}$ vibrational frequency values, which MUST have been correctly
ordered according to the calculated frequencies, are defined in free format. For example,
\begin{Verbatim}[frame=single]
 $ExpFrq $End
 835.0248 835.3904 926.0930 926.2148
 2160.9759
\end{Verbatim}

If \verb|MODE = 1|, a list of theoretical frequencies will be replaced by
the provided experimental ones. One frequency per line, including the
frequency index and its experimental value. The program will
read experimental frequencies until a blank line or the end is encountered.
For example,
\begin{Verbatim}[frame=single]
 $ExpFrq MODE=1 $End
 3  926.0930
 5 2160.9759
\end{Verbatim}


\subsection{\texttt{\$Thermo} group} \label{sec:inp-thermo}
\index{{\texttt{\$Thermo} group}@{\texttt{\$Thermo} group}}
\index{{Thermochemistry}@{Thermochemistry}}

This group controls the thermochemistry calculation. Keywords:

\bigskip{}
\index{{\texttt{\$Thermo} group}@{\texttt{\$Thermo} group}!\texttt{Eel}}
\verb|Eel|: total energy taken from quantum chemistry calculation (in Hartree). Default: 0.

\bigskip{}
\index{{\texttt{\$Thermo} group}@{\texttt{\$Thermo} group}!\texttt{NDeg}}
\verb|NDeg|: degeneracy of the (spin-orbit) electronic state, which affects the entropy and Gibbs free energy. Default: 1.

\bigskip{}
\index{{\texttt{\$Thermo} group}@{\texttt{\$Thermo} group}!\texttt{Temp}}
\verb|Temp|: temperature (in K). Default: 298.15.

\bigskip{}
\index{{\texttt{\$Thermo} group}@{\texttt{\$Thermo} group}!\texttt{Press}}
\verb|Press|: pressure (in atm). Default: 1.0.

\bigskip{}
\index{{\texttt{\$Thermo} group}@{\texttt{\$Thermo} group}!\texttt{Scale}}
\index{{\texttt{\$ExpFrq} group}@{\texttt{\$ExpFrq} group}}
\index{{Experimental frequency correction}@{Experimental frequency correction}}
\verb|Scale|: frequency scale factor. Default: 1.0. Experimental frequencies defined in the \texttt{\$ExpFrq} group will not be scaled.

\bigskip{}
\index{{\texttt{\$Thermo} group}@{\texttt{\$Thermo} group}!\texttt{PG}}
\index{{Symmetry}@{Symmetry}}
\verb|PG|: specify the name of point group to calculate rotational
entropy. It may affect the entropy and Gibbs free energy, so the correct point group
name must be provided.
\begin{description}
\item[ ]\verb|  = 0|: (default) 2.
\item[ ]\verb|  = 1|: use the isotope independent point group. If isotope leads to lower symmetry, this option may
reproduce the results of other quantum chemistry programs, but unfortunately this is
not correct.
\item[ ]\verb|  = 2|: use the isotope dependent point group.
\item[ ]\verb|  = "XXXX"|: specify the name of point group, which is useful for the high symmetry not supported by the program, for example, \verb|"D10h"|. Don't forget the quotes.
\end{description}


\subsection{Atomic thermochemistry calculation} \label{sec:inp-atom}
\index{{Thermochemistry}@{Thermochemistry}!{Atomic thermochemistry}}

Atomic thermochemistry data can also be calculated using the \textsc{UniMoVib} program, which are useful to
study some atom related chemical reactions, for example, $CH_3 + H_2 \rightarrow CH_4 + H$. The atomic
thermochemistry calculation does not require any data and files from quantum chemistry calculations except the
optional total energy. Three groups of keywords may be provided:

\begin{description}
\index{{\texttt{\$Contrl} group}@{\texttt{\$Contrl} group}!\texttt{QCProg}}
\item[ ]\verb|  $Contrl| group (section \ref{sec:inp-contrl}): \verb|qcprog="atomcalc"| should be specified.
\index{{\texttt{\$IsoMas} group}@{\texttt{\$IsoMas} group}}
\index{{\texttt{\$Contrl} group}@{\texttt{\$Contrl} group}!\texttt{Isotop}}
\item[ ]\verb|  $IsoMas| group (section \ref{sec:inp-isomas}): the atomic mass should be specified (\verb|Isotop| in the \verb|$Contrl| group is always 2).
\index{{\texttt{\$Thermo} group}@{\texttt{\$Thermo} group}!\texttt{Eel}}
\index{{\texttt{\$Thermo} group}@{\texttt{\$Thermo} group}!\texttt{NDeg}}
\index{{\texttt{\$Thermo} group}@{\texttt{\$Thermo} group}!\texttt{Temp}}
\index{{\texttt{\$Thermo} group}@{\texttt{\$Thermo} group}!\texttt{Press}}
\item[ ]\verb|  $Thermo| group (section \ref{sec:inp-thermo}): \verb|Eel|, \verb|NDeg|, \verb|Temp|, and \verb|Press| can be specified, which are optional.
\end{description}
The other namelists and keywords do not make sense and will be ignored. See Example \ref{sec:exp1}.

\pagebreak{}


\section{Examples} \label{part:examp}
\index{{Example}@{Example}}

\subsection{Atomic thermochemistry calculation} \label{sec:exp1}
\index{{Example}@{Example}!{Atomic thermochemistry}}
\index{{Thermochemistry}@{Thermochemistry}!{Atomic thermochemistry}}

\begin{Verbatim}[frame=single,label=example,labelposition=topline,numbers=left,rulecolor=\color{green},fontsize=\footnotesize,baselinestretch=1.0]
Atomic thermochemistry calculation (Ne atom)
The total energy was calculated at the HF/STO-3G level

 $contrl
   qcprog="atomcalc"
 $end

 $Thermo
   Eel=-127.8038245 Temp=500 Press=10
 $end

 $IsoMas $End
 19.99244
\end{Verbatim}

\subsection{Frequency calculation by \textsc{Gaussian}} \label{sec:exp2}
\index{{Example}@{Example}!{Gaussian}}
\index{{Quantum chemistry program}@{Quantum chemistry program}!Gaussian}

\begin{Verbatim}[frame=single,label=example,labelposition=topline,numbers=left,rulecolor=\color{green},fontsize=\footnotesize,baselinestretch=1.0]
a test job

 $contrl
   qcprog="gaussian"
 $end

 $qcdata
   fchk="xef6.fchk"
 $end
\end{Verbatim}

\subsection{Frequency calculation by \textsc{Molpro}} \label{sec:exp3}
\index{{Example}@{Example}!{Molpro}}
\index{{Quantum chemistry program}@{Quantum chemistry program}!Molpro}

\textsc{Molpro} cannot handle non-Abelian point group symmetries, like $T_d$ in CH$_4$. Using \textsc{UniMoVib}, you can get \emph{irreps.} of normal vibrational modes in full symmetry.

\begin{Verbatim}[frame=single,label=example,labelposition=topline,numbers=left,rulecolor=\color{green},fontsize=\footnotesize,baselinestretch=1.0]
a test job

 $contrl
   qcprog="molpro"
 $end

 $qcdata
   fchk="ch4.out"
 $end
\end{Verbatim}

\subsection{Calculate ``experimental'' frequencies of HDO} \label{sec:exp4}
\index{{Example}@{Example}!{``Experimental'' frequencies of HDO}}
\index{{UniMoVib format}@{UniMoVib format}}

One can estimate frequencies of HDO from experimental frequencies of H$_2$O.

\begin{Verbatim}[frame=single,label=example,labelposition=topline,numbers=left,rulecolor=\color{green},fontsize=\footnotesize,baselinestretch=1.0]
Step 1.
Save data file using experimental frequencies of H2O.
The normal modes should be calculated at high level of theory.

 $contrl
   qcprog="cfour"
   ifsave=.true.
   ifexp=.true.
 $end

 $qcdata
   fchk="h2o.out"
   geom="GRD"
 $end

 $expfrq mode=1 $end
 1   1595
 2   3657
 3   3756   B2
\end{Verbatim}

\begin{Verbatim}[frame=single,label=example,labelposition=topline,numbers=left,rulecolor=\color{green},fontsize=\footnotesize,baselinestretch=1.0]
Step 2.
Calculate "experimental" frequencies of HDO using the experimental frequencies of H2O.
CCSD(T)/cc-pVTZ frequencies:        1463  2828  3895 cm-1
"experimental" frequencies:         1398  2692  3708 cm-1
real experimental frequencies:      1402  2727  3707 cm-1

 $contrl
   qcprog="unimovib"
   isotop=1
 $end

 $qcdata
   fchk="step1.umv"
 $end

 $IsoMas $End
 2  2.01410
\end{Verbatim}


\pagebreak{}


\section{Known problems} \label{part:problem}
\index{{Known problems}@{Known problems}}
\index{{Quantum chemistry program}@{Quantum chemistry program}}

\begin{itemize}
\index{{Symmetry}@{Symmetry}}
\item Open-shell molecule.

Because of (pseudo-)Jahn-Teller effects, maybe the \textit{irreps.} cannot be recognized by the program.

\index{{Quantum chemistry program}@{Quantum chemistry program}!ADF}
\item \textsc{Adf}

For a numerical frequency calculation (for example, by two-component ZORA): if symmetry is used by \textsc{Adf}, and if
there are symmetry-equivalent atoms, the IR intensities cannot be calculated correctly since some elements in the APT matrix are
missing. The correct values can be obtained from \textsc{Adf} output if you are interested in the IR intensities.

\index{{Quantum chemistry program}@{Quantum chemistry program}!MOLDEN}
\item \textsc{Molden}

\begin{enumerate}
\item If there are imaginary frequencies, some program may print positive frequency values
by mistake (for example, \textsc{CFour}), then the results will be wrong. So you have to
check the imaginary frequencies in the \textsc{Molden} file, and correct them manually
if the negative sign is missing.

\item Since there is no isotopic mass information in the \textsc{Molden} file, the most
abundant isotopic masses are assumed. If this is not true, however, the results
will be totally wrong. So you have to use the most abundant isotopic masses
in your frequency calculation to generate a \textsc{Molden} file.
\end{enumerate}

\index{{Quantum chemistry program}@{Quantum chemistry program}!ACES}
\item \textsc{Aces}

Since there is no isotopic mass information in \textsc{Aces} output, the most
abundant isotopic masses are assumed. If this is not true, however, the results
will be totally wrong. So you have to use the most abundant isotopic masses
in your frequency calculation (this is default in \textsc{Aces}) to generate a \textsc{Aces} output file.

\end{itemize}


\pagebreak{}

\appendix
\section{Appendix} \label{part:appdx}

\subsection{Format of the UniMoVib data file} \label{sec:almfmt}
\index{{UniMoVib format}@{UniMoVib format}}

The UniMoVib data file is an ASCII one in free format.
\begin{Verbatim}[frame=single,label=Format(Ver.1.0.1 2017.10.15),labelposition=topline,rulecolor=\color{green},fontsize=\small,baselinestretch=1.0]
(One title line)
NATM
  (An positive integer)
AMASS
  (NATM number of atomic masses)
ZA
  (NATM number of nuclear charge numbers)
XYZ
  (3*NATM elements of Cartesian coordinates in a.u.;
  Use XYZANG instead if in Angstrom)
FFX
  (3NATM*3NATM elements of Hessian matrix;
  Use FFXLT instead for L.T. matrix)
APT
  (3*3NATM elements of APT data;
  Use NOAPT instead if no APT data provided)
DPR
  (6*3NATM elements of polarizability derivatives;
  Use DPRSQ instead if in the form of 9*3NATM;
  Use NODPR instead if no DPR data provided)
\end{Verbatim}
DPR data and Raman intensities are not supported at present.

\pagebreak{}

\phantomsection
\addcontentsline{toc}{part}{Index}
\printindex

\end{document}
